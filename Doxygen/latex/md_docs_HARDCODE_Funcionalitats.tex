\subsection*{Comprimir i descomprimir fitxers txt}

Ho fa mitjançant algoritmes LZ.

Si l\textquotesingle{}usuari escull el mode manual pot triar quin algorisme emprar entre els L\+Z78, L\+Z\+SS i L\+ZW.

Si escull el mode automàtic el nostre programa tria l\textquotesingle{}algorisme en funció de la mida del fitxer\+: L\+Z\+SS per a fitxers de menys de 1\+MB i L\+ZW per a fitxers més grans. Aquesta heurística és nova d\textquotesingle{}aquesta entrega ja que a la primera entrega el mode automàtic sempre comprimia amb L\+ZW.

\subsection*{Comprimir i descomprimir imatges ppm}

Ho fa mitjançant l\textquotesingle{}algoritme J\+P\+EG.

A més, si l\textquotesingle{}usuari escull mode manual, pot triar la qualitat amb què es comprimeix. Com més qualitat menys grau de compressió i com menys qualitat més grau de compressió.

Si l\textquotesingle{}usuari escull mode automàtic la qualitat serà del 50\%.

La funcionalitat de triar la qualitat de compressió és nova respecte la primera entrega

\subsection*{Comrpimir i descomprimir qualsevol fitxer (funcionalitat nova)}

Si el fitxer que l\textquotesingle{}usuari vol comprimir no és ni ppm ni txt, es comprimirà amb un algoritme LZ, de la mateixa manera que si fos txt.

En descomprimir-\/se, recuperarà l\textquotesingle{}extensió original gràcies a la metadata del comprimit.

\subsection*{Comprimir i descomprimir carpetes (funcionalitat nova)}

També es poden comprimir i descomprimir carpetes. Per cada fitxer que contingui\+:
\begin{DoxyItemize}
\item Si és ppm serà comrpimit amb l\textquotesingle{}algoritme J\+P\+EG amb la qualitat que demani l\textquotesingle{}usuari (en cas de compressió manual) o amb una qualitat del 50\% (en cas de compressió automàtica)
\item Altrament, serà comrpimit amb un algoritme LZ, el que especifiqui l\textquotesingle{}usuari (en cas de compressió manual) o el que decideixi el programa mitjançant l\textquotesingle{}heurística explicada més amunt (en cas de compressió automàtica).
\end{DoxyItemize}

\subsection*{Visualitzar fitxers de text (funcionalitat nova)}

El nostre programa permet visualitxar fitxer de text. Si el fitxer de text que es volvisualitzar està comprimit, es descomprimeix automàticament (en un fitxer temporal) per poder ser mostrat a la interfície gràfica d\textquotesingle{}usuari.

\subsection*{Extensió única (funcionalitat nova)}

Tots els comprimits generats pel nostre programa tenen la mateixa extensió\+: .jm, a diferència de la primera entrega, en què l\textquotesingle{}extensió representava l\textquotesingle{}algorisme amb què s\textquotesingle{}havia comprimit. Ara el programa sap amb quin algorisme s\textquotesingle{}ha comprimit gràcies a la metadata del comprimit. 