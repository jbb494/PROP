\section*{Manual de Usuario}

Cuando abramos el ejecutable veremos la siguiente interfície\+:



\subsection*{Secciones de la interficíe}

\subsubsection*{1. Opción de previsualización}

Si tenemos la casilla marcada cuando tenemos seleccionado un fichero comprimido podremos observar como se abre una visualización que muestra como quedará el archivo trás una descompresión.

\subsubsection*{2. Seleccionar fichero/directorio}

Al pinchar sobre el icono de la carpeta se abrirá un panel donde podremos escoger cualquier archivo visible de nuestro directorio, cuya dirección será mostrada en la barra de \char`\"{}path\char`\"{}.



\subsubsection*{3. Barra de path}

Una vez hayamos seleccionado el archivo con el que queremos trabajar, su ruta será mostrada aquí.

\subsubsection*{4. Compresión Automática}

Una vez tengamos un fichero seleccionado, si no se trata de un archivo ya comprimido, podremos comprimirlo de manera automática al hacer click. Será el propio programa quien se encargue de tomar todas las decisiones.

\subsubsection*{5. Compresión Manual}

De la misma manera que en la {\bfseries compresión Automática}, una vez tengamos seleccionado un fichero (a excepción de los ya comprimidos) podremos hacer click en esta opción. Una vez hecho se nos abrirá una ventana donde podremos escoger como queremos que sea tratado nuestro archivo durante el proceso.



Por defecto viene seleccionado un {\itshape grado de compresión bajo} para los archivos {\itshape {\bfseries .ppm}} (los cuales serán procesados con {\itshape J\+P\+EG}) y el algoritmo {\itshape L\+ZW} para el resto de archivos. En caso de que queramos modificar la decisión simplemente tendremos que escoger lo que deseemos y darle a {\bfseries Aceptar}.

{\itshape Nota para $\ast$$\ast$.ppm$\ast$$\ast$\+: Cuánto más grande sea el grado de compresión menos memoria ocupará la imagen comprimida pero menos calidad tendrá al ser descomprimida.}

\subsubsection*{6. Descompresión}

De la misma manera que en la compresión, una vez tengamos seleccionado un archivo (en este caso únicamente podrá ser {\itshape {\bfseries .ppm}}) se nos habilitará la opción de descomprimirlo.

A diferencia de la compresión, únicamente se nos proporcionará una opción automática, dado que será el propio programa quien se encargue de decidir el algoritmo para la descompresión según el que hayamos utilitzado para la compresión.

\subsubsection*{7. Salir}

Una vez hayamos acabado de realizar todo lo que necesitásemos, no tenemos más que hacer click sobre este botón para que el programa finalice y se cierre.

 

\subsection*{Funcionamiento}

Para empezar a usar el programa basta con seguir los pasos siguientes.

Lo primero que tendremos que hacer siempre será seleccionar el archivo con el que queramos trabajar (directorios incluídos). Una vez seleccionado el fichero podremos observar en nuestra {\bfseries barra de path} su ruta.

{\itshape Si por algún error no hemos seleccionado el fichero deseado simplemente tenemos que empezar este paso desde el inicio.}

Una vez tengamos nuestro archivo seleccionado, tendremos varias posibilidades según el tipo de fichero\+:


\begin{DoxyEnumerate}
\item {\itshape {\bfseries Fichero comprimido}}\+: En este caso se nos habilitará la opción de descomprimirlo. En el supuesto caso de que no estemos seguros de que sea el archivo que queremos descomprimir o por cualquier otra razón, tendremos también la opción de habilitar la casilla {\bfseries Visualizar} y se nos mostrará el fichero sin dejarlo guardado en el correspondiente directorio.
\item {\itshape {\bfseries Fichero descomprimido}}\+: En este caso podremos escoger entre una compresión manual o automática. Si escogemos la opción manual simplemente tenemos que seguir los pasos detallados en su correspondiente apartado (explicado en la sección anterior).
\end{DoxyEnumerate}

Una vez seleccionada nuestra opción, en el caso de que haya sido compresión/descompresión una vez haya acabado el proceso se nos mostrarán las estadistícas, como bien pueden ser el grado de compresión/descompresión o el tiempo que haya tardado.

Una vez acabado el proceso podemos volver a empezar de 0 seleccionando un nuevo archivo o salir del programa.

\subsection*{Aclaraciones}


\begin{DoxyItemize}
\item Una vez tengamos seleccionado un archivo y lo comprimamos o descomprimamos, el resultado generado será guardado en el mismo directorio donde se encuentre el original.
\item Si nos disponemos a descomprimir un fichero con su original (o uno con el mismo nombre) en ese mismo directorio, es importante tener en cuenta que el nuevo fichero {\bfseries sobreescribirá} al anterior.
\item Hay que tener en cuenta que J\+P\+EG, el algoritmo usado para comprimir imágenes .ppm es lossy, es decir, pierde información. No obstante, si se selecciona la opción de grado de compresión bajo se puede conseguir una compresión casi loseless, es decir, que la pérdida de información es tan pequeña que el ojo humano no la puede percibir.
\begin{DoxyItemize}
\item Para imágenes pequeñas (alrededor de 100 kB) la pérdida de información con grado de compresión alto es muy notable y con grado medio es apreciable.
\item Para imágenes grandes (alrededor de 1 MB) la pérdida de información con grado de compresión alto es poco apreciable y con grado medio es casi inapreciable. 
\end{DoxyItemize}
\end{DoxyItemize}