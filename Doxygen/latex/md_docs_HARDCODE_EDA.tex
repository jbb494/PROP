\subsection*{Dict\+\_\+\+Encode}

Estructura de dades on cada posició representarà una seqüencia de bytes b0,b1,...,bn-\/1,bn. Per entendre com es genera i s\textquotesingle{}obté aquesta seqüencia es recomanable mirar la descripció de la classe {\itshape Node} donat que és l\textquotesingle{}estructura de dades de cada posició del nostre Array.

Les propietats principals son les següents.

\subsubsection*{reset\+\_\+dictionary()}

S\textquotesingle{}encarrega de reinicialitzar el diccionari, el qual passarà a tenir 256 posicions (codi A\+S\+C\+II). Només el cridarem quan el diccionari sigui ple (de manera automàtica) o en la construcció del qual.

\subsubsection*{Ascii\+\_\+value(byte c)}

Retorna el valor númeric corresponent al byte c.

\subsubsection*{search\+\_\+and\+\_\+insert\+\_\+\+B\+S\+T(\+Integer i, byte c)}

És la funció principal de la classe i s\textquotesingle{}encarrega de les següents funcions\+: $>$En primer cas mira si cal reiniciar el diccionari, en aquest cas ho fa directament sense haver de cancel·lar tot el procés, etc.

$>$Cada cop que es crida es mira si ja tenim la cadena al diccionari indicada per (i,c).~\newline
 -\/\+En cas afirmatiu retornem la seva posició.~\newline
 -\/\+Si teniem el prefix però no la cadena, busquem exactament on s\textquotesingle{}insereix.~\newline
 -\/\+En cas contrari, l\textquotesingle{}afegim al diccionari.

\subsection*{Dict\+\_\+\+Decode}

Estructura de dades on cada posició indicarà una seqüencia de bytes b0,b1,...,bn-\/1,bn. El primer valor de l\textquotesingle{}Array indica la posició del següent byte de la subcadena b0,...,bn-\/1 mentre que el segon valor és bn.

\subsubsection*{reset\+\_\+dictionary(\+Boolean b)}

S\textquotesingle{}encarrega de reinicialitzar el diccionari, el qual passarà a tenir 256 posicions (codi A\+S\+C\+II) només en cas de que b sigui true. Només el cridarem quan el diccionari sigui ple (de manera automàtica) o en la construcció del qual.

\subsubsection*{get\+Word(\+Integer i)}

S\textquotesingle{}encarrega de recorrer de manera iterativa l\textquotesingle{}Array de la classe fins que el valor de i sigui -\/1. Amb això el que aconseguim és retornar la paraula que correspon a la seqüència encadenada de bytes a partir de i.

\subsection*{Pair}

És un template de la classe Pair de C++ el qual ha sigut creat per no haver d\textquotesingle{}utilitzar llibreries externes de Java.

\subsection*{Node}

Representa un node emprat per la representació de cadenes de caràcters, utilitzat en la classe L\+ZW.

Consta de 4 atributs de classe\+:
\begin{DoxyItemize}
\item First -\/$>$ Serà l\textquotesingle{}index al primer node que utilitzi la cadena de bytes que representi aquest.
\item c -\/$>$ Valor del node
\item Left -\/$>$ Index dels següents nodes(x) que utilitzin el mateix prefix que aquest i on x $<$= c
\item Right -\/$>$ Index dels següents nodes(x) que utilitzin el mateix prefix que aquest i on x $>$ c
\end{DoxyItemize}

\subsection*{Array\+Circular}

Té la mateixa idea que una Array, però no hi ha cap remove. I, si s\textquotesingle{}excedeix el size del array, s\textquotesingle{}anira sobreescrivint els valors del inici i així seqüencialment amb un punter (anomenat start). És a dir, utilitza una política circular. Té diferents funcions\+:

\subsubsection*{set\+Value(byte value)}

Que afegeix al array circular el valor value

\subsubsection*{get\+Value(int index)}

Que retorna el valor que es troba a la posició index de l\textquotesingle{}array circular. En el cas que es trobi fora de rang, es crea una excepció.

\subsubsection*{get\+Value\+Amb\+Despl(int despl)}

Utilitzat pel descompressor. El qual va despl vegades enrere des de la posició del punter, retornant el valor en aquella posició.

\subsubsection*{is\+In(byte value)}

Retorna un boolean indicant si value es troba o no a l\textquotesingle{}array circular. També hi ha operacions simples com retornar la posició del punter a la primera posició o el punter a la última posició, el nombre de valors afegits, que, com es obvi, té un màxim del valor que se li ha pasat per parametre al constructor, el qual serà la mida de la array.

\subsection*{Intor\+Byte}

En aquesta classe es tracta de construir un objecte per obtenir una unitat mínima per al descompressor. El constructor admetrà com a paràmetre un boolean, que indicarà si és un Byte (true) o la tupla despl-\/mida (false). Per tant, té les operacions òbvies de set i get de els atributs byte, despl i mida. Per últim, també té la consultora de si és un byte o dos Int\textquotesingle{}s.

\subsection*{Trie}

Aquesta classe és una implementació general de l\textquotesingle{}estructura de dades Trie. Cada node representa una seqüencia de Bytes. Cada connexió entre nodes (pare-\/fill), representa un caràcter. I un node representa la seqüència de caràcters des d\textquotesingle{}ell fins a l\textquotesingle{}arrel. A més, cada node té un enter que identifica la seva seqüència. (això serà útil per al compressor). Tot i que aquesta classe està implementada de forma general (), parlarem com si fos un Trie, ja que és com l\textquotesingle{}utilitzarem sempre en aquest projecte.

\subsubsection*{insert(Array\+List list, Integer índex)}

L\textquotesingle{}algoritme lz78 només fa {\bfseries insert}, si en el diccionari ja existeix la frase {\itshape list} -\/ (últim Byte de {\itshape list}). O sigui que mai es donarà el cas que la funció inserti més d\textquotesingle{}un node al Trie. Dit això, aquesta funció afegeix un Node (que serà fill del node que representi {\itshape list} -\/ (últim Byte de {\itshape list})), i li assigna el {\itshape index}.

\subsubsection*{index\+Node(\+Array\+List list)}

Retorna l\textquotesingle{}índex del node que representa la frase {\itshape list}. Retorna -\/1 si no existeix.

\subsection*{Trie\+Node}

Aquesta estructura de dades són els nodes del Trie. Cada node representa la seqüencia de bytes des d\textquotesingle{}ell fins a l\textquotesingle{}arrel. L\textquotesingle{}atribut {\itshape index} representa l\textquotesingle{}índex del node (que és l\textquotesingle{}identificador d\textquotesingle{}aquest). Després tenim {\itshape children}, que es un Hash\+Map que cada entrada té com a key un byte i com a value un Trie\+Node. Després tenim dos consultores, una que retorna el Has\+Map que conté {\itshape children}. La segona constructora retorna l\textquotesingle{}índex d\textquotesingle{}aquest mateix Trie\+Node. Per últim, el constructor, que se li passa com a paràmetre l\textquotesingle{}índex.

\subsection*{Bin\+Tree}

Bin\+Tree representa un arbre binari, on només les fulles poden tenir valor. Per tant, permet representar un arbre de \hyperlink{classHuffman}{Huffman}.

El primer que se’m va acudir per representar un arbre binari és que contingués tres atributs\+: el valor de l’arrel, si en té; el subarbre fill esquerra, si en té; i el subarbre fill dret. Tanmateix em preocupava l’eficiència espacial de recórrer un arbre, ja que l’única manera que se m’acudia d’obtenir informació d’un node x és obtenir el subarbre del qual x és arrel i, per tant, haver de replicar tots els descendents de x.

Per tant vaig buscar una alternativa\+: assignaria un enter que identifiqués cada node i l’accés a un node amb al seu identificador seria directe\+: es podria obtenir en cost constant sabre si és fulla, saber el seu valor (en cas de ser fulla) i sabre l’identificador dels seus fills.

L’atribut principal de Bin\+Tree és arr, que funciona de la següent manera\+: sigui x un node si arr\mbox{[}2x\mbox{]} = -\/2, x és una fulla i arr\mbox{[}2x+1\mbox{]} és el valor de x altrament x+arr\mbox{[}2x\mbox{]} i x+arr\mbox{[}2x+1\mbox{]} són els fills de x (-\/1 significa indefinit) Un node no està inicialitzat $<$-\/$>$ els dos fills son indefinits El node arrel és 0

Per afegit un fill a un node identificat per x passem per paràmetre el subarbre que inclou el node fill a afegir amb tots els seus descendents. Com que la informació dels fills s’emmagatzema a l’atribut arr com a distància relativa entre l’índex del pare i l’índex del fill, no cal modificar l’arr del subarbre i només cal afegir-\/lo al final de l’arr de l’arbre al qual afegim i assignar a arr\mbox{[}2x\mbox{]} (si és fill esquerre) o a arr\mbox{[}2x+1\mbox{]} (si és fill dret) la distància relativa entre pare i fill.

\section*{Algoritmes}

\subsection*{L\+ZW (Lempel–\+Ziv–\+Welch, 1984)}

El mètode L\+ZW crea sobre la marxa, de manera automàtica i en una única passada, un diccionari de cadenes que es trobin dins del text a comprimir mentre al mateix temps es procedeix a la seva codificació. Aquest diccionari no és transmès amb el text comprimit, ja que el descompressor pot reconstruir-\/lo usant la mateixa lògica amb què el fa el compressor i, si està codificat correctament, tindrà exactament les mateixes cadenes que el diccionari del compressor tenia.

\subsubsection*{Compressió}

En la compressió amb L\+ZW s\textquotesingle{}ha emprat l\textquotesingle{}estructura de dades {\itshape Dict\+\_\+\+Encode} (la qual ha estat definida a l\textquotesingle{}apartat de {\bfseries Estructures de Dades}). En un principi es va utilitzar un {\itshape Map$<$Integer,String$>$} i això feia que la compressió fos massa lenta. En aquest moment i tenint en compte els dos factors següents, es va prendre la decisió d\textquotesingle{}eliminar els Strings i simplificar la representació de les cadenes\+:
\begin{DoxyItemize}
\item No necessitem emmagatzemar la cadena sencera de Strings.
\item Qualsevol nou String que tinguem es pot representar com un append d\textquotesingle{}un byte al String anterior. Ara mateix ens quedariem doncs amb un {\itshape Map$<$Integer,Byte$>$} on cada entrada representa l\textquotesingle{}index a la seqüència previa i el byte actual, inicialitzat amb (-\/1,$\ast$) on $\ast$ són els primers 256 valors A\+S\+C\+II .
\end{DoxyItemize}

En aquest moment el compressor funcionava relativament bé però (inspirat en l\textquotesingle{}article de Juha Nieminen), es va decidir substituir el map per crear una estructura de dades propia que millores el funcionament, {\itshape Dict\+\_\+\+Encode}. La definició de la qual, com bé s\textquotesingle{}ha dit al començament, ja està explicada més amunt, però la idea per voler utilitzar-\/la es que d\textquotesingle{}aquesta manera podiem buscar si una paraula ja estava al diccionari o en cas contrari afegir-\/la en una mateixa crida, sense haver de recorrer el contingut dos cops. A més, amb la representació d\textquotesingle{}arbre aconseguim agilitzar el procés donat que tindrem un byte ordenat amb tots els altres que accedeixin a un mateix prefixe.

D\textquotesingle{}aquesta manera el que farem es per cada byte que llegim del fitxer d\textquotesingle{}entrada, buscarem si la cadena que representa ja existeix, en aquest cas llegirem el següent amb la intenció de generar una cadena més gran, en cas contrari afegim aquesta cadena al diccionari i passem al byte següent.

~\newline


~\newline


~\newline


\subsubsection*{Descompressió}

En la descompressió amb L\+ZW hem emprat l\textquotesingle{}estructura de dades {\itshape Dict\+\_\+\+Decode} (la qual ha estat definida a l\textquotesingle{}apartat de {\bfseries Estructures de Dades}). En un primer moment es va utilitzar un {\itshape Array\+List$<$\+String$>$} però aixó no era eficient per unes raons molt semblants a les de la compressió\+:
\begin{DoxyItemize}
\item No necessitem emmagatzemar tota l\textquotesingle{}estona el text descomprimit.
\item Podem anar ajuntant els bytes que anem descomprimint amb els que ja estaven descomprimits.
\end{DoxyItemize}

Per aixó es van eliminar els Strings pel que es va quedar l\textquotesingle{}estructura de dades {\itshape Array\+List$<$ Pair$<$Integer,Byte$>$ $>$} amb la qual només necessitariem anar unint els bytes segons l\textquotesingle{}index al que apunta cada qual.

D\textquotesingle{}aquesta manera el que farem es anar generant de nou un diccionari a mesura que anem descomprimint cada byte i retornant la cadena que representa cada decodificació.

\subsection*{L\+Z\+SS (Lempel–\+Ziv–\+Storer–\+Szymansk, 1982)}

En aquesta classe es tracta la compressió mitjançant l\textquotesingle{}algorisme L\+Z\+SS i la descompressió d\textquotesingle{}un input el qual ha estat comprimit amb aquest mateix algoritme. Per la part de la compressió es té una finestra corredissa de 2048 posicions (11 bits), ja que s\textquotesingle{}han fet proves estimant un equilibri eficiència/temps. Després la mida de 5 bits, per tant, es buscarà una compressió mínima de 3B. Per la part de la descompressió, s\textquotesingle{}agafarà la unitat mínima representada per Intor\+Byte, on, com bé especifica aquella classe, seran dos Int\textquotesingle{}s o un Byte.

\subsubsection*{Compressió}

Es crean dos estructures de dades\+: Array\+Circular, per la finestra corredissa i Array\+List, per la subparaula que es voldrà trobar a la finestra. ~\newline
 L\textquotesingle{}Array\+List és de Bytes per la qüestió que s\textquotesingle{}ha de tractar utf-\/8, ja que si es posa de char\textquotesingle{}s, java farà una interpretació A\+S\+C\+II. ~\newline
 En el cas de l\textquotesingle{}Array\+Circular s\textquotesingle{}ha utilitzat per reduir el cost a nivell de temps del programa, ja que, anteriorment, s\textquotesingle{}utilitzava un Tree\+Map$<$Integer, Byte$>$ el qual, quan arribava al size màxim de la finestra, s\textquotesingle{}anava fent remove de les primeres posicions afegides (molt costós). Per tant, amb un array amb política circular, no cal fer cap remove i reestructurar (molt menys costós). ~\newline
 A continuació, es fa un while, on, a cada iteració, s\textquotesingle{}agafa un byte del input. Amb aquest byte nou, s\textquotesingle{}afegeix primerament a l\textquotesingle{}estructura de dades Array\+List (subparaula). Aquesta, es buscarà si està o no a la finestra, mitjançant l\textquotesingle{}algorisme Knuth-\/\+Morris-\/\+Pratt, amb cost lineal. Per tant, es tenen dos casos\+: ~\newline
 $>$S\textquotesingle{}ha trobat a la finestra\+: ~\newline
 \begin{quote}


\begin{quote}
$>$Llavors es continuarà la busqueda d\textquotesingle{}una subparaula més gran a la finestra afegint bytes a aquesta (amb un màxim de mida de 34, ja que, com s\textquotesingle{}utilitzen 5 bits de mida, 2⁵-\/1 == 31 (rang), llavors, se sap que 0, 1 i 2 no s\textquotesingle{}utilitzarà, per una qüestió que no es comprimiran bytes de mida més petita a 3, es pot afegir +3 a la mida per comprimir). ~\newline
 \end{quote}
\end{quote}


\begin{quote}
No s\textquotesingle{}ha trobat a la finestra\+: ~\newline
 \begin{quote}
$>$Llavors, si la subparaula, es compleix que mida-\/1 $>$= 3 (-\/1 ja que al inici s\textquotesingle{}afegeix un byte a aquesta subparaula que no s\textquotesingle{}ha trobat), s\textquotesingle{}haurà d\textquotesingle{}afegir al Output el substring de la subparaula (de la posició 0 a size-\/2) i continuar buscant \char`\"{}match\textquotesingle{}s\char`\"{} per l\textquotesingle{}últim byte afegit a la subparaula.~\newline
 \end{quote}
\end{quote}
En el cas que no es compleixi la condició anterior, s\textquotesingle{}haurà d\textquotesingle{}afegir el byte sense comprimir al Output.

\subsubsection*{Descompressió}

Es crea una estructura de dades Array\+Circular, per representar la finestra corredissa ja introduïda, la qual, s\textquotesingle{}anirà omplint a mesura que estem al while agafant byte\textquotesingle{}s del input. També es crea un Intor\+Byte, ja introduït a apartats anteriors. Dins del while, s\textquotesingle{}aniran agafant instàncies Intor\+Byte, les quals, representaran, o bé un byte i per tant, s\textquotesingle{}afegeix a la finestra i al output, o bé dos Int\textquotesingle{}s, els quals seran una tupla punter/tamany, que s\textquotesingle{}aniran afegint al Output i a la vegada a la finestra.

\subsection*{L\+Z78 (Lempel-\/\+Ziv, 1978)}

L\+Z78 és un algoritme {\itshape lossless}, i que en temps lineal comprimeix un arxiu qualsevol. L\textquotesingle{}estructura de dades que utilitza és un diccionari, el qual és una col·lecció potencialment il·limitada de frases vistes anteriorment. Cada frase en el diccionari té associada un índex que codificarà aquesta frase.

\subsubsection*{Compressió}

La compressió té el següent funcionament\+: A mesura que va llegint, va introduïnt frases a un diccionari, i després, quan es troba una repetició d\textquotesingle{}alguna frase que estigui al diccionari, {\itshape imprimeix} l\textquotesingle{}índex de l\textquotesingle{}entrada del diccionari en lloc de la frase. En la meva implementació de l\textquotesingle{}algoritme, vaig començar utilitzant un Hash\+Map$<$\+String,\+Integer$>$;. La clau era la frase, i el valor era l\textquotesingle{}índex que representava aquella frase en concret. L\textquotesingle{}algoritme funcionava bé, a una bona velocitat (seguia sent lineal). Però utilitzava molt espai de memòria (R\+AM). Vaig decidir canviar el Hash\+Map per un altre tipus de diccionari\+: El Trie. L\textquotesingle{}aventatge que té el Trie respecte al Hash\+Map és que, tot i mantenir la funció {\itshape find} a cost temporal constant, utilitza molta menys memòria. Ja que cada frase emmagatzemada al Trie ocupa sempre 1 byte, mentre que el Hash\+Map ocupava tants bytes com la mida de la frase. Això es deu a l\textquotesingle{}estructura que segueix el Trie, que cada node representa la seqüència de bytes que va des d\textquotesingle{}ells fins a l\textquotesingle{}arrel.

\subsubsection*{Descompressió}

El text descomprimit, per la natura de la compressió, és sempre una seqüència de Pair$<$\+Integer,\+Byte$>$;. Tenint això en compte, la descompressió segueix aquest funcionament\+: Anem generant un nou diccionari a mesura que llegim l\textquotesingle{}entrada. A més, els integers que llegim són entrades al diccionari que existeixen (a no ser que sigui 0, que aleshores no està al diccionari i hem d\textquotesingle{}afegir-\/lo). Quan llegim un integer, {\itshape imprimim} la frase apuntada per aquest. Els bytes els {\itshape imprimim} tal com els llegim.

L\textquotesingle{}estructura de dades emperada a la descompressió ha estat {\itshape Dict\+\_\+\+Decode} (la qual ha estat definida a l\textquotesingle{}apartat de {\bfseries Estructures de Dades}).

D\textquotesingle{}aquesta manera el que farem es anar generant de nou un diccionari a mesura que anem descomprimint cada byte i retornant la cadena que representa cada decodificació.

\subsection*{J\+P\+EG (Joint Photographic Experts Group)}

Explicaré simultàniament la compressió i la descompressió ja que els passos d’un son els inversos dels passos de l’altre en ordre invers.

\subsubsection*{Block splitting}

El fet d’haver de treballar en blocs 8x8 fa que el compressor processa els píxels en el mateix ordre en què es llegeixen i el descompressor no processa els píxels en el mateix ordre en què s’escriuen. Això és així perquè en una imatge P\+PM entre una fila de píxels d’un bloc i la següent hi ha n files d’altres blocs, on n = (amplada de la imatge)/8. Per això la unitat mínima de lectura i escriptura d’una imatge P\+PM per a l’algoritme J\+P\+EG és el conjunt dels n blocs que comparteixen files entre ells, concretament una matriu nx8x8x3, diguem-\/ne mat, on mat\mbox{[}b\mbox{]}\mbox{[}i\mbox{]}\mbox{[}j\mbox{]}\mbox{[}k\mbox{]} és la quantitat de color \#k (color \#0 és red; color \#1 és green; color \#2 és blue) del píxel (i, j) del bloc b.

En el compressor, un cop llegida aquesta matriu, ja es pot processar cadascun dels n blocs per separat. En el descompressor un cop processats cadascun dels n blocs per separat ja es pot escriure aquesta matriu. Les classes Ctrl\+\_\+\+Input\+\_\+\+Img i Ctrl\+\_\+\+Output\+\_\+\+Img s’encarreguen de llegir i escriure, respectivament, en el format d’aquesta matriu.

\subsubsection*{Color space transformation}

Sigui (r, g, b) la tripleta que representa un píxel en format R\+GB Sigui (y, cb, cr) la tripleta que representa un píxel en format Y\+Cb\+Cr.

Per transformar de R\+GB a Y\+Cb\+Cr en un rang de 0 a 255 he utilitzat les següents assignacions\+: y = 16 + (65.\+738r + 129.\+057g + 25.\+064b)/256; cb = 128 + (-\/37.\+945r -\/ 74.\+494g + 112.\+439b)/256; cr = 128 + (112.\+439r -\/ 94.\+154g -\/ 18.\+285b)/256;

Per transformar de Y\+Cb\+Cr a R\+GB en un rang de 0 a 255 he utilitzat les següents assignacions\+: r = 255./219.(y-\/16) + 255./224. 1.\+402 (cr-\/128.); g = 255./219.(y-\/16) + 255./224. (1.\+772 0.\+114/0.587 (cb-\/128) -\/ 1.\+402 0.\+299/0.587 (cr-\/128)); b = 255./219.(y-\/16) + 255./224. 1.\+772 (cb-\/128);

Vaig voler comprovar el bon funcionament de la transformació fent el següent\+: per diversos colors R\+GB, transformar-\/los a Y\+Cb\+Cr, tornar-\/los a transformar a R\+GB i comparant els valors inicials de R\+GB amb els finals. Vaig veure que no eren exactament les mateixes però que les peites diferències eren imperceptibles a l’ull humà a l’hora de visualitzat el color. Tanmateix era preocupant que alguns valors sortissin del rang (0, 255) ja que això no permetria representar-\/los bé en el fitxer ppm. Per això en després d’aplicar la transformació de Y\+Cb\+Cr si algun valor era negatiu forçava que fos 0 i si algun valor sobrepassava 255 forçava que fos 255.

\subsubsection*{Downsampling}

Per aquesta entrega no he aplicat downsampling, ja que vaig llegir que no era un pas necessari. Tot i que estic satisfet amb el factor de compressió que he aconseguit sense fer downsampling em plantejaré fer-\/ne per la segona entrega.

\subsubsection*{Discrete cosine transformation}

Per cada canal Y\+Cb\+Cr vaig aplicar D\+CT en la compressió i D\+CT inversa en la descompressió.

\subsubsection*{Quantization}

La matriu de quantització que he utilitzat és la següent\+:

\{ \{16, 11, 10, 16, 24, 40, 51, 61\}, \{12, 12, 14, 19, 26, 58, 60, 55\}, \{14, 13, 16, 24, 40, 57, 69, 56\}, \{14, 17, 22, 29, 51, 87, 80, 62\}, \{18, 22, 37, 56, 68, 109, 103, 77\}, \{24, 35, 55, 64, 81, 104, 113, 92\}, \{49, 64, 78, 87, 103, 121, 120, 101\}, \{72, 92, 95, 98, 112, 100, 103, 99\} \};

La vaig trobar a \href{https://en.wikipedia.org/wiki/JPEG#Quantization}{\tt https\+://en.\+wikipedia.\+org/wiki/\+J\+P\+E\+G\#\+Quantization} , on explica que genera comprimits amb una qualitat d’un 50\%.

Provoca un factor de compressió notable i una qualitat acceptable per imatges grans. Tot i que per imatges petites (d’un centenar de píxels d’alçada i amplada) hi falta qualitat.

Per la segona entrega faré que es pugui reglar la qualitat.

\subsubsection*{Entropy coding}

La codificació de \hyperlink{classHuffman}{Huffman} se n’encarrega una classe específica, que genera un codi de \hyperlink{classHuffman}{Huffman} per cada possible valor que pugui prendre el que jo anomeno “símbol 1 de entropy coding”, proporciona un codi de \hyperlink{classHuffman}{Huffman} d’una mida determinada. També fa la transformació inversa\+: per cada codi de \hyperlink{classHuffman}{Huffman} proporciona el un valor pel símbol 1 de entropy coding. El que jo anomeno “símbol 1 de entropy coding” és un byte que conté la rinlength als 4 bits de més pes i la size (mida del símbol 2 de entropy coding) als 4 bits de menys pes, tal com explica \href{https://en.wikipedia.org/wiki/JPEG#Entropy_coding}{\tt https\+://en.\+wikipedia.\+org/wiki/\+J\+P\+E\+G\#\+Entropy\+\_\+coding} .

Un atribut de la classe J\+P\+EG és una instància de la classe \hyperlink{classHuffman}{Huffman}.

El símbol 2 de la entropy coding depèn del valor de la posició en qüestiò de la matriu que resulta de la quantització i té mida variable. He fet aquesta conversió de la manera que explica la següent web\+: \href{https://www.impulseadventure.com/photo/jpeg-huffman-coding.html}{\tt https\+://www.\+impulseadventure.\+com/photo/jpeg-\/huffman-\/coding.\+html} .

Per la descompressió, com que el símbol 2 té mida variable cal anar llegint del fitxer .jpeg bit a bit i anar preguntant a la instància de la classe \hyperlink{classHuffman}{Huffman} si pels bits llegits fins al moment s’ha trobat un símbol pel codi format pels bits llegits fins al moment i seguir llegint bits fins a trobar-\/lo.

\subsection*{\hyperlink{classHuffman}{Huffman}}

Aquesta classe se’n carrega de de proporcionar la relació de codis de \hyperlink{classHuffman}{Huffman} per una sèrie de símbols.

Per aquesta primera entrega he optat per utilitzar una taula de \hyperlink{classHuffman}{Huffman} predeterminada, especialment pensada per l’algoritme J\+P\+EG. És la que vaig trobar en el següent link\+: \href{https://www.ece.ucdavis.edu/cerl/reliablejpeg/coding/}{\tt https\+://www.\+ece.\+ucdavis.\+edu/cerl/reliablejpeg/coding/} .

Per la segona entrega em plantejaré fer dependre la taula de \hyperlink{classHuffman}{Huffman} de les freqüències amb què apareixen els símbols.

La classe \hyperlink{classHuffman}{Huffman} té un atribut que el vector auto\+\_\+codes. Donat un símbol x, Donat un enter x, code(auto\+\_\+codes\mbox{[}x\mbox{]}) és la codificació de x, que té una mida de size(auto\+\_\+codes\mbox{[}x\mbox{]}). size(long) i code(long) són funcions estàtiques i privades de la classe.

Un altre atribut és tree, que és l’arbre de \hyperlink{classHuffman}{Huffman} i és de tipus Bin\+Tree, una classe especialment creada per ser utilitzada per \hyperlink{classHuffman}{Huffman}.

La constructora de la classe té coma paràmetre un booleà que indica si \hyperlink{classHuffman}{Huffman} serà automàtic o manual. De moment, només funciona el mode automàtic, ja que per aquesta entrega he fet que la taula de \hyperlink{classHuffman}{Huffman} sigui predeterminada. A la constructora s’inicialitza el vector auto\+\_\+codes amb els valors de la taula de \hyperlink{classHuffman}{Huffman} predeterminada i es genera tree en funció de auto\+\_\+codes.

Per obtenir un codi de \hyperlink{classHuffman}{Huffman} a partir d’un símbol, es fa amb les funcions get\+Code(int symbol) i get\+Size(int symbol) i és tan fàcil com consultat la posició symbol del vector auto\+\_\+codes.

Si tenim una tira se bits amb un codi de \hyperlink{classHuffman}{Huffman} i volem obtenir el símbol que representen podem utilitzar get\+Symbol(int code) però com que possiblement no sabrem on acaba el codi i no volem llegir més bits dels necessaris podem seguir el següent procés\+:

init\+Search\+Symbol() mentre no haguem trobat el símbol o tinguem la certesa que el codi no és vàlid\+: llegim un bit executem search\+Symbol(int bit), que ens indicarà si s’ha trobat el símbol, cal seguir buscant o el codi no és vàlid si s’ha trobat un símbol l’obtenim amb la funció get\+Found\+Symbol()

Aquest procés funciona gràcies a un atribut privat de la classe que identifica un node de l’arbre de \hyperlink{classHuffman}{Huffman} i fuciona com a punter. 