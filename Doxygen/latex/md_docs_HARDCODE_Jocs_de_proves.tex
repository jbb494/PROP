Introduirem a continuació una petita descripció de cada joc de proves, per a la compressió i descompressió, amb l\textquotesingle{}objectiu rere el contingut de cada fitxer.

Els arxius de Input\+X.\+txt estan a la carpeta src/persistencia/data

Per a cada arxiu X\+X\+X.\+txt existeix una copia del qual anomenada X\+X\+X\+Check.\+txt, la qual farem servir per comprovar que el resultat obtingut de la compressió -\/$>$ descompressió és el desitjat.

{\itshape Recomanem el comandament \char`\"{}diff X\+X\+X.\+txt X\+X\+X\+Check.\+txt\char`\"{} de Linux per comprovar el correcte funcionament del codi.}

\subsection*{Fitxers txt}

\subsubsection*{Input1.\+txt}

Es tracta d\textquotesingle{}un fitxer de 4.\+4 MB en el que es repeteix un cert paràgraf un nombre determinat de vegades. Ens centrem en intentar maximitzar el grau de compressió de l\textquotesingle{}arxiu.

\subsubsection*{Input2.\+txt}

En aquest cas estem treballant amb un fitxer de 5.\+5 MB, amb el que busquem veure si la velocitat de compressió dels nostres algorismes es veu greument perjudicada amb un augment de la mida o si es manté dintre d\textquotesingle{}uns marges raonables degut a que l\textquotesingle{}algorisme en qüestió segueix el grau de complexitat esperat.

\subsubsection*{Input3.\+txt}

Tornem a tenir un fitxer de menys de 1.\+5 MB (1.\+1 MB concretament), però en aquest cas no estem repetint de manera intencionada el seu contingut. La intenció es veure com es comporten els nostres algorismes amb un text de mida raonable.

\subsubsection*{Input4.\+txt}

Vam afegir un tercer text de mida similar al Input1/3 per comprovar si el grau de compressió es mantenia per sobre de 1 en textos de tamany superior a 1 MB.

\subsubsection*{Input5.\+txt}

En tots els casos anteriors estàvem treballant sobre textos en català/castellà, fet que no ens acabava de convèncer per assegurar el correcte funcionament de la compressió/descompressió de textos utf8, per tant vam generar un fitxer amb codificació Koreana.

\subsubsection*{Input6.\+txt}

A últim moment ens va sorgir el dubte sobre qué passaria si el fitxer estava completament buit, així que vam crear aquest sisé joc de proves.

\subsection*{Junit}

\subsubsection*{junit.\+txt -\/ junit.\+lzw}

Emprem aquests dos fitxers per la comprovació de la classe L\+ZW amb Junit. 