Introduirem a continuació una petita descripció de cada joc de proves, per a la compressió i descompressió, amb l\textquotesingle{}objectiu rere el contingut de cada fitxer.

\subsection*{Fitxers txt}

Els arxius de Input\+X.\+txt estan a la carpeta src/persistencia/data

Per a cada arxiu X\+X\+X.\+txt existeix una copia del qual anomenada X\+X\+X\+Check.\+txt, la qual farem servir per comprovar que el resultat obtingut de la compressió -\/$>$ descompressió és el desitjat. Cal tenir en compte que si comprimim X\+X\+X.\+txt se\textquotesingle{}ns genera X\+X\+X.\+jm. I si després comprimim X\+X\+X.\+jm se sobreescriu X\+X\+X.\+txt.

{\itshape Recomanem la comanda \char`\"{}diff X\+X\+X.\+txt X\+X\+X\+Check.\+txt\char`\"{} de Linux per comprovar el correcte funcionament del codi.}

\subsubsection*{Input1.\+txt}

Es tracta d\textquotesingle{}un fitxer de 4.\+4 MB en el que es repeteix un cert paràgraf un nombre determinat de vegades. Ens centrem en intentar maximitzar el grau de compressió de l\textquotesingle{}arxiu.

\subsubsection*{Input2.\+txt}

En aquest cas estem treballant amb un fitxer de 5.\+5 MB, amb el que busquem veure si la velocitat de compressió dels nostres algorismes es veu greument perjudicada amb un augment de la mida o si es manté dintre d\textquotesingle{}uns marges raonables degut a que l\textquotesingle{}algorisme en qüestió segueix el grau de complexitat esperat.

\subsubsection*{Input3.\+txt}

Tornem a tenir un fitxer de menys de 1.\+5 MB (1.\+1 MB concretament), però en aquest cas no estem repetint de manera intencionada el seu contingut. La intenció es veure com es comporten els nostres algorismes amb un text de mida raonable.

\subsubsection*{Input4.\+txt}

Vam afegir un tercer text de mida similar al Input1/3 per comprovar si el grau de compressió es mantenia per sobre de 1 en textos de tamany superior a 1 MB.

\subsubsection*{Input5.\+txt}

En tots els casos anteriors estàvem treballant sobre textos en català/castellà, fet que no ens acabava de convèncer per assegurar el correcte funcionament de la compressió/descompressió de textos utf8, per tant vam generar un fitxer amb codificació Koreana.

\subsubsection*{Input6.\+txt}

A últim moment ens va sorgir el dubte sobre qué passaria si el fitxer estava completament buit, així que vam crear aquest sisé joc de proves.

\subsection*{Imatges ppm}

A src/persistencia/data/imatges\+\_\+grans hi ha imatges d\textquotesingle{}entre 1,4 MB i 2,3 MB.

A src/persistencia/data/imatges\+\_\+petites hi ha imatges d\textquotesingle{}entre 12 kB i 232 kB.

Per a cada arxiu X\+X\+X.\+ppm existeix una copia del qual anomenada X\+X\+X\+C\+H\+E\+C\+K.\+ppm, la qual farem servir per comparar que el resultat obtingut de la compressió -\/$>$ descompressió és el desitjat. Cal tenir en compte que si comprimim X\+X\+X.\+ppm se\textquotesingle{}ns genera X\+X\+X.\+jm. I si després comprimim X\+X\+X.\+jm se sobreescriu X\+X\+X.\+ppm.

Com que J\+P\+EG és lossy les imatges no es poden comparar amb la comanda {\itshape diff} sinó que s\textquotesingle{}ha de fer a ull.

\subsection*{Carpetes}

src/persistencia/data/carpeta és un exemple de carpeta bastant complet pensada per ser comprimida. Conté\+:
\begin{DoxyItemize}
\item {\itshape house.\+ppm}, una imatge petita
\item {\itshape sage.\+ppm}, una imatge gran
\item {\itshape subcarpeta}, una carpeta amb\+:
\begin{DoxyItemize}
\item {\itshape curt.\+txt}, un fitxer de text curt
\item {\itshape llarg.\+txt}, un fitxer de text llarg
\item {\itshape max.\+cc}, el codi d\textquotesingle{}un programa en C++. Serà comprimit amb un algoritme LZ, igual que els fitxers txt.
\end{DoxyItemize}
\end{DoxyItemize}

src/persistencia/data/carpeta\+C\+H\+E\+CK és una còpia de la carpeta per comprovar que el procés compressió-\/$>$descompressió s\textquotesingle{}hagi fet correctament.

\subsection*{Junit}

Els quatre fitxers que hi ha a src/persistencia/data/junit (junit.\+jm, junit.\+txt, junit\+\_\+check.\+jm i junit\+\_\+check.\+txt) son emprats per la comprovació de la classe L\+ZW amb Junit. Per tal que funioni la comprovació no han de ser modificats per cap actor que no sigui el programa encarregat de fer la comprovació. 