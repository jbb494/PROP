Es tracta de construir un entorn per a la compressió i descompressió de fitxers .txt i de imatges .ppm

\section*{Compressor i Descompressor}

\subsection*{Fitxers .T\+XT}

\subsubsection*{Algorismes}

\begin{quote}
\paragraph*{\href{./classdomini_1_1algorithm_1_1LZ78.html}{\tt L\+Z78 }\+: Lempel-\/\+Ziv, 1978}

\paragraph*{\href{./classdomini_1_1algorithm_1_1LZSS.html}{\tt L\+Z\+SS }\+: Lempel–\+Ziv–\+Storer–\+Szymansk, 1982}

\paragraph*{\href{./classdomini_1_1algorithm_1_1LZW.html}{\tt L\+ZW }\+: Lempel–\+Ziv–\+Welch, 1984}

\end{quote}
\subsection*{Imatges .P\+PM}

\subsubsection*{Algorisme}

\begin{quote}
\paragraph*{\href{./classdomini_1_1algorithm_1_1JPEG.html}{\tt J\+P\+EG }\+: Joint Photographic Experts Group, 1992}

\end{quote}
~\newline


\section*{Com utilitzar l\textquotesingle{}instal·lador}

\begin{quote}
Ens anem a la alçada de la carpeta de src. Allà, obrim un terminal i executem \char`\"{}./compile\+And\+Run.\+sh\char`\"{}. Llavors, ens sortirà un menú desplegable amb les opcions disponibles, com aquest\+: \begin{quote}
make compila el projecte.~\newline
 run per executar el main.~\newline
 -\/driver i el nom del driver que vols executar. Exemple\+: ./compile\+And\+Run.sh -\/driver \hyperlink{classdomini_1_1algorithm_1_1Driver____LZ78}{domini.\+algorithm.\+Driver\+\_\+\+\_\+\+L\+Z78}~\newline
 -\/ctest Compila els tests.~\newline
 -\/etest i el nom del test que vols executar. Exemple\+: ./compile\+And\+Run.sh -\/etest \hyperlink{classdomini_1_1algorithm_1_1LZWTest}{domini.\+algorithm.\+L\+Z\+W\+Test}~\newline
 -\/clean o -\/remove Borra els executables. \end{quote}
\end{quote}
~\newline
 ~\newline
 Treball realitzat per\+: \begin{quote}
Manel Aguilar~\newline
 Joan Bellavista~\newline
 Joan Lapeyra~\newline
 Miguel Paracuellos~\newline
\end{quote}
